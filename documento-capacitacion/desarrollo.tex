\section{Capacitación para el personal de logística}

	\subsection{Necesidad}
		La empresa Transportes Jaguar ha sucumbido ante el tema de la desactualización y se ha unido al mundo de las nuevas técnicas, es por ello que se encuentra en la necesidad de formar un equipo más y mejor capacitado en el área de transporte de mercancías. En la actualidad existen más de ciento quince mil empresas de autotransporte en sus distintas clasificaciones (hombre-camión, pequeña, mediana y grande) de las cuales podemos destacar (Transportes Castores, TUMSA, Transportes Quintanilla, etc) entre ellas se encuentra el personal más calificado con equipos en funcionamiento de última generación, tomando en cuenta estos datos importantes debemos actuar para cumplir uno de los objetivos más importantes de la empresa "brindar un servicio de calidad".
		
		\subsection{Dessarollo de solución}
		
Se planea contratar una capacitación para el área de logística, la encargada de gestionar los recursos necesarios para el desempeño del servicio de transporte brindado a nuestros clientes, además y lo más importante, realizar un plan de ejecución el cual analiza cómo desarrollar la metodología a estudiar  y ponerla en marcha con el menor gasto posible. Con el punto anterior sustentamos dos puntos importantes:
	\begin{itemize}
		\item Tener un equipo capacitado en técnicas de último impacto en el mercado.
		\item Ejecutar el plan al mercado con el menor gasto posible.
	\end{itemize}
	
La capacitación tendrá una duración de 1 mes con un total de 24 horas de clase, 1 hora de clase por día, ésta se llevará a cabo en el mes de Octubre; el encargado de la capacitación es la \textbf{Fundación de Investigación para el Desarrollo Profesional con el Programa de Distribución y Logística en México}. Su costo es de \textbf{\$5,715}, el objetivo del curso es ''utilizar los elementos fundamentales del proceso de distribución y logística en la toma de decisiones vinculada con la cadena de suministro, para la identificación de asuntos relacionados con el abastecimiento de productos y el nivel de servicio en las organizaciones y su impacto en las ventajas competitivas de la empresa''. Al final de la capacitación se realizará un examen, el cual además de comprobar lo que se ha aprendido, se otorgará un diploma de valor curricular de la Secretaría del Trabajo y Previsión Social.

	\subsection{Plan de trabajo}
	\textbf{Tareas que se ejecutaran a lo largo de la capacitación:\\\\}
	\textit{Análisis de lo que se está haciendo hoy. (Antecedentes y conceptos)}
		\subsubsection{Primer semana}
			\begin{enumerate}
				\item LA LOGÍSTICA Y LA CADENA DE SUMINISTRO.
				\begin{itemize}
					\item Configuración de la empresa.
					\item Conceptos logísticos.
					\item Cadena de suministro.
					\item Tendencias en las operaciones, almacenamiento y distribución.
				\end{itemize}
				\item MODELO DE PLANEACIÓN MAESTRA DE RECURSOS.
				\begin{itemize}
					\item Planeación estratégica.
					\item Administración de la Demanda y pronósticos.
					\item Servicio al cliente.
					\item Proceso de Plan de Ventas y Operaciones.
					\item Proceso de Programación Maestra.
				\end{itemize}
			\end{enumerate}
		\textit{ \\Metodologías probadas para incrementar lo que se hace bien, de ser el caso contrario tomar medidas de re-ingenieria de procesos.}
		\subsubsection{Segunda semana}	
		 	\begin{enumerate}
				\item SISTEMAS DE DISTRIBUCIÓN
				\begin{itemize}
					\item Sistema de empujar (push). Sistema de jalar (pull).
					\item Sistema cuello de botella.
					\item Sistemas colaborativos.
				\end{itemize}
			\end{enumerate}
		\textit{ \\Relaciones y buen trato con proveedores y clientes.}
		\subsubsection{Tercer semana}	
		\begin{enumerate}
				\item ESTRATEGIA DE ABASTECIMIENTOS
				\begin{itemize}
					\item Principios de relacionamiento con los proveedores.
					\item Abastecimiento táctico VS abastecimiento estratégico.
					\item Evaluación de clientes.
					\item Desarrollo de clientes.
					\item Sistemas colaborativos.
				\end{itemize}
			\end{enumerate}
		\textit{\\Formalizar procesos de transporte de mercancía, implementando herramientas y mecanismos para hacer lo mayor deseado con lo menor posible.}
		\subsubsection{Cuarta semana}	
		\begin{enumerate}
				\item ESTRATEGIA DE TRANSPORTE
				\begin{itemize}
					\item Fundamentos del transporte.
					\item Selección de servicios de transporte
					\item Diseño de rutas (costo-recursos).
				\end{itemize}
			\end{enumerate}		
			
	\subsection{ Justificación}
	Al terminar la distribución y obtención de conocimientos e información  antes mencionada al personal de logística, tendremos que poner en práctica lo aprendido diseñando nuevas estrategias de convencimiento y abastecimiento de necesidades a nuestros clientes, tomando como comparativa nuestros resultados en tiempos / ingresos del ejercicio fiscal pasado. Hemos de saber de antemano que el mercado se sentirá mucho más satisfecho al fortalecer nuestras políticas de calidad y por consiguiente incrementaremos los bien formados lazos de negocios formados con nuestros más reconocidos clientes.
	\subsection{Impacto}
	Se muestran los siguientes elementos que se verán alterados al poner en marcha nuestro plan de mejora continua.
	\begin{itemize}
		\item Mejor servicio, con el menor costo posible.
		\item Mayores ingresos, al tener mayores y más grandes clientes.
		\item Mejora continua de personal y maquinaria, con los ingresos obtenidos.
		\item Cadena de distribución y abastecimiento fortalecida.
		\item  Incremento y esparcimiento de la empresa (Fomar un nuevo punto de encuentro).
		\item Dar a conocer al usuario final la efectividad y compromiso de Transportes Jaguar.
	\end{itemize}